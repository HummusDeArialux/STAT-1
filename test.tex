% Options for packages loaded elsewhere
\PassOptionsToPackage{unicode}{hyperref}
\PassOptionsToPackage{hyphens}{url}
%
\documentclass[
]{article}
\usepackage{amsmath,amssymb}
\usepackage{lmodern}
\usepackage{iftex}
\ifPDFTeX
  \usepackage[T1]{fontenc}
  \usepackage[utf8]{inputenc}
  \usepackage{textcomp} % provide euro and other symbols
\else % if luatex or xetex
  \usepackage{unicode-math}
  \defaultfontfeatures{Scale=MatchLowercase}
  \defaultfontfeatures[\rmfamily]{Ligatures=TeX,Scale=1}
\fi
% Use upquote if available, for straight quotes in verbatim environments
\IfFileExists{upquote.sty}{\usepackage{upquote}}{}
\IfFileExists{microtype.sty}{% use microtype if available
  \usepackage[]{microtype}
  \UseMicrotypeSet[protrusion]{basicmath} % disable protrusion for tt fonts
}{}
\makeatletter
\@ifundefined{KOMAClassName}{% if non-KOMA class
  \IfFileExists{parskip.sty}{%
    \usepackage{parskip}
  }{% else
    \setlength{\parindent}{0pt}
    \setlength{\parskip}{6pt plus 2pt minus 1pt}}
}{% if KOMA class
  \KOMAoptions{parskip=half}}
\makeatother
\usepackage{xcolor}
\usepackage[margin=1in]{geometry}
\usepackage{listings}
\newcommand{\passthrough}[1]{#1}
\lstset{defaultdialect=[5.3]Lua}
\lstset{defaultdialect=[x86masm]Assembler}
\usepackage{graphicx}
\makeatletter
\def\maxwidth{\ifdim\Gin@nat@width>\linewidth\linewidth\else\Gin@nat@width\fi}
\def\maxheight{\ifdim\Gin@nat@height>\textheight\textheight\else\Gin@nat@height\fi}
\makeatother
% Scale images if necessary, so that they will not overflow the page
% margins by default, and it is still possible to overwrite the defaults
% using explicit options in \includegraphics[width, height, ...]{}
\setkeys{Gin}{width=\maxwidth,height=\maxheight,keepaspectratio}
% Set default figure placement to htbp
\makeatletter
\def\fps@figure{htbp}
\makeatother
\setlength{\emergencystretch}{3em} % prevent overfull lines
\providecommand{\tightlist}{%
  \setlength{\itemsep}{0pt}\setlength{\parskip}{0pt}}
\setcounter{secnumdepth}{-\maxdimen} % remove section numbering
\ifLuaTeX
\usepackage[bidi=basic]{babel}
\else
\usepackage[bidi=default]{babel}
\fi
\babelprovide[main,import]{spanish}
% get rid of language-specific shorthands (see #6817):
\let\LanguageShortHands\languageshorthands
\def\languageshorthands#1{}
\usepackage{listings}
\usepackage{xcolor}

\definecolor{codegreen}{rgb}{0,0.6,0}
\definecolor{codeblue}{rgb}{0,0,0.6}
\definecolor{codegray}{rgb}{0.5,0.5,0.5}
\definecolor{codepurple}{rgb}{0.58,0,0.82}
\definecolor{backcolour}{rgb}{0.95,0.95,0.92}

\lstdefinestyle{mystyle}{
backgroundcolor=\color{backcolour},   
commentstyle=\color{codegreen},
keywordstyle=\color{magenta},
numberstyle=\color{codeblue},
stringstyle=\color{codepurple},
breakatwhitespace=false,         
breaklines=true,                 
captionpos=b,                    
keepspaces=true,                 
showspaces=false,                
showstringspaces=false,
showtabs=false,                  
tabsize=2
}

\lstset{style = mystyle}

\ifLuaTeX
  \usepackage{selnolig}  % disable illegal ligatures
\fi
\IfFileExists{bookmark.sty}{\usepackage{bookmark}}{\usepackage{hyperref}}
\IfFileExists{xurl.sty}{\usepackage{xurl}}{} % add URL line breaks if available
\urlstyle{same} % disable monospaced font for URLs
\hypersetup{
  pdftitle={test},
  pdfauthor={Maria Lucas},
  pdflang={Es-es},
  hidelinks,
  pdfcreator={LaTeX via pandoc}}

\title{test}
\author{Maria Lucas}
\date{2023-04-22}

\begin{document}
\maketitle

{
\setcounter{tocdepth}{4}
\tableofcontents
}
\hypertarget{ejercicio-1}{%
\section{Ejercicio 1}\label{ejercicio-1}}

Primero, cargamos los datos del documento excel.

\begin{lstlisting}[language=R]
#install.packages("readxl")
library("readxl")
data1 = read_excel("cicindela.xlsx")
names(data1)[1] <- "BD"
names(data1)[2] <- "WE"
names(data1)[3] <- "SPS"
names(data1)[4] <- "BS"
names(data1)[5] <- "AD"
\end{lstlisting}

\hypertarget{ajuste-del-modelo}{%
\subsubsection{Ajuste del modelo}\label{ajuste-del-modelo}}

LA IA ME DICE QUE CHECKEE LAS ASUNCIONES (lineal, independencia,
homocedasticidad, N de residuos)

\begin{lstlisting}[language=R]
# Creación del modelo
lmod = lm(BD ~ WE + SPS + BS + AD, data = data1)
sum = summary(lmod)
# Estadístico anova
anv = anova(lmod)
sum
\end{lstlisting}

\begin{lstlisting}
## 
## Call:
## lm(formula = BD ~ WE + SPS + BS + AD, data = data1)
## 
## Residuals:
##     Min      1Q  Median      3Q     Max 
## -6.3004 -2.7038  0.0795  2.6017  5.3924 
## 
## Coefficients:
##             Estimate Std. Error t value Pr(>|t|)  
## (Intercept)  14.9531    17.2661   0.866   0.4152  
## WE            0.9123     1.0935   0.834   0.4317  
## SPS           3.8970     1.1690   3.334   0.0125 *
## BS            0.6511     0.4530   1.437   0.1938  
## AD           -1.5624     0.6610  -2.364   0.0501 .
## ---
## Signif. codes:  0 '***' 0.001 '**' 0.01 '*' 0.05 '.' 0.1 ' ' 1
## 
## Residual standard error: 4.513 on 7 degrees of freedom
## Multiple R-squared:  0.9578, Adjusted R-squared:  0.9337 
## F-statistic: 39.71 on 4 and 7 DF,  p-value: 6.727e-05
\end{lstlisting}

Como podemos observar mediante la estimación de los coeficientes de
regresión, la ecuación quedaría como: BD = 14.95 + 0.91WE + 3.89SPS +
0.65BS - 1.56AD.

El modelo obtenido es significativo, con un pvalor global = 6.727e-05.
El test estadístico empleado es un F-test, éste testa como H0 que todos
los coeficientes de regresión son 0, y como H1 que al menos uno es
distinto de 0.

H0: B1 = B2 = \ldots{} = Bp = 0 (donde B1, B2, \ldots, Bp son los
coeficientes de regresión de las variables predictoras del modelo) H1:
al menos un Bi es diferente a 0, donde i = 1, 2, \ldots, p

En este caso al menos una de las variables tiene dependencia lineal con
la variable respuesta (Beetle Density).

\end{document}
